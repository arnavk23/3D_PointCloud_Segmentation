\documentclass[12pt,a4paper]{report}
\usepackage[utf8]{inputenc}
\usepackage[T1]{fontenc}
\usepackage{times}
\usepackage{geometry}
\usepackage{fancyhdr}
\usepackage{titlesec}
\usepackage{tocloft}
\usepackage{amsmath,amsfonts,amssymb}
\usepackage{graphicx}
\usepackage{booktabs}
\usepackage{longtable}
\usepackage{array}
\usepackage{listings}
\usepackage{xcolor}
\usepackage{hyperref}
\usepackage{cite}
\usepackage{url}
\usepackage{setspace}
\usepackage{caption}
\usepackage{subcaption}
\geometry{left=1.5in,right=1in,top=1in,bottom=1in,headheight=14.5pt}
\onehalfspacing
\pagestyle{fancy}
\fancyhf{}
\fancyhead[R]{\thepage}
\fancyhead[L]{\leftmark}
\renewcommand{\headrulewidth}{0.4pt}
\titleformat{\chapter}[display]{\normalfont\huge\bfseries\centering}{\chaptertitlename\ \thechapter}{20pt}{\Huge}
\titlespacing*{\chapter}{0pt}{-20pt}{40pt}
\titleformat{\section}{\normalfont\Large\bfseries}{\thesection}{1em}{}
\titleformat{\subsection}{\normalfont\large\bfseries}{\thesubsection}{1em}{}
\lstset{language=Python,basicstyle=\ttfamily\footnotesize,keywordstyle=\color{blue},commentstyle=\color{green!60!black},stringstyle=\color{red},numbers=left,numberstyle=\tiny\color{gray},stepnumber=1,numbersep=8pt,showstringspaces=false,breaklines=true,frame=tb,framerule=0.5pt,xleftmargin=15pt,framexleftmargin=15pt,framextopmargin=6pt,framexbottommargin=6pt,captionpos=b}
\hypersetup{colorlinks=true,linkcolor=black,filecolor=magenta,urlcolor=blue,citecolor=black,pdftitle={Condensed 3D Point Cloud Semantic Segmentation Report},pdfauthor={Arnav Kapoor}}
\title{Condensed 3D Point Cloud Semantic Segmentation Report}
\author{Arnav Kapoor}
\date{Summer 2025}
\begin{document}
% Title Page
\begin{titlepage}
    \centering
    \vspace*{1cm}
    {\LARGE\textbf{3D Point Cloud Semantic Segmentation using Deep Learning}}
    \vspace{0.5cm}
    {\Large Condensed Internship Report}
    \vspace{2cm}
    {\large Submitted by}\\
    \vspace{0.5cm}
    {\Large\textbf{Arnav Kapoor}}
    \vspace{2cm}
    \includegraphics[width=3cm]{logo.png}
    \vspace{1cm}
    {\large Under the supervision of}\\
    \vspace{0.5cm}
    {\Large\textbf{Prof. Vaibhav Kumar}}
    \vspace{2cm}
    {\large GeoAI4Cities Laboratory, IISER Bhopal}
    \vspace{2cm}
    {\large Summer 2025}
    \vfill
    {\large \textbf{Hardware Used:} NVIDIA Jetson, ZED 2i Camera, NVIDIA RTX 3070}
\end{titlepage}
% Abstract
\newpage
\chapter*{Abstract}
\addcontentsline{toc}{chapter}{Abstract}
This condensed report summarizes research on 3D point cloud semantic segmentation using deep learning, optimized for edge computing platforms. Four models (PointNet, SONATA, PVCNN, RandLA-Net) were implemented and evaluated for real-time segmentation on NVIDIA Jetson devices. Key contributions include model optimization, memory management, and deployment pipeline development. Results show SONATA achieves highest accuracy, while PointNet offers best speed-resource balance. The work advances practical edge AI for robotics and smart cities.
\textbf{Keywords:} Point Cloud Segmentation, Edge Computing, Deep Learning, Real-time Inference, Model Optimization
% Table of Contents
\newpage
\tableofcontents
\newpage
% Introduction
\chapter{Introduction}
\section{Background}
Point cloud segmentation enables machines to classify 3D environments for autonomous systems. This work bridges deep learning advances and edge deployment constraints.
\section{Objectives}
\begin{enumerate}
    \item Implement and optimize four deep learning models for 3D segmentation
    \item Deploy models on NVIDIA Jetson platforms
    \item Develop real-time pipeline from ZED 2i camera to segmentation
    \item Evaluate performance across hardware
\end{enumerate}
\section{Significance}
Efficient edge deployment is critical for robotics, autonomous navigation, and urban monitoring. This internship provided hands-on experience in practical AI engineering.
% Literature Review
\chapter{Literature Review}
\section{Point Cloud Deep Learning}
Summarizes key models: PointNet, PointNet++, PVCNN, Dynamic Graph CNN, SONATA, RandLA-Net. Reviews recent surveys and benchmarks.
\section{Edge Computing for 3D Vision}
Discusses challenges and solutions for deploying deep models on resource-limited devices. Highlights Jetson platforms and optimization techniques.
\section{Datasets and Benchmarks}
ShapeNet, S3DIS, SemanticKITTI, Semantic3D used for training and evaluation.
% Methodology
\chapter{Methodology}
\section{Development Environment}
Describes hardware (Jetson, RTX 3070, ZED 2i) and software stack (PyTorch, Open3D, ZED SDK, TensorRT).
\section{Model Implementations}
Briefly outlines architectures and code structure for each model. Key optimizations: mixed precision, quantization, memory management.
\section{Pipeline}
End-to-end workflow from data capture to segmentation and visualization.
% Results and Analysis
\chapter{Results and Analysis}
\section{Performance Metrics}
Presents accuracy, speed, and memory usage for each model. Table summarizes results.
\begin{table}[htbp]
\centering
\begin{tabular}{lcccc}
\toprule
Model & mIoU (\%) & Acc (\%) & FPS (Jetson) & Mem (GB) \\
\midrule
PointNet & 73.2 & 89.1 & 28.7 & 0.8 \\
PVCNN & 78.6 & 91.4 & 16.4 & 1.9 \\
SONATA & 81.4 & 93.2 & 14.9 & 2.4 \\
RandLA-Net & 76.9 & 90.7 & 20.5 & 1.4 \\
\bottomrule
\end{tabular}
\caption{Model performance comparison on Jetson NX}
\end{table}
\section{Optimization Impact}
Summarizes effects of mixed precision, quantization, and TensorRT on speed and memory.
\section{Deployment}
Describes real-time integration and validation on Jetson devices.
% Discussion
\chapter{Discussion}
\section{Challenges}
Addresses memory constraints, CUDA errors, and deployment hurdles.
\section{Lessons Learned}
Highlights practical engineering insights from edge AI deployment.
% Conclusion
\chapter{Conclusion}
This condensed report demonstrates practical deployment of 3D deep learning models on edge platforms. Key results: real-time segmentation, efficient memory use, and robust performance. The work contributes to democratizing advanced AI for smart cities and robotics.
% References
\newpage
\bibliographystyle{plain}
\bibliography{references}
\end{document}
